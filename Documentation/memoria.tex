\documentclass[a4paper,12pt,twoside]{memoir}

% Castellano
\usepackage[spanish,es-tabla]{babel}
\selectlanguage{spanish}
\usepackage[utf8]{inputenc}
\usepackage[T1]{fontenc}
\usepackage{lmodern} % Scalable font
\usepackage{microtype}
\usepackage{placeins}

\RequirePackage{booktabs}
\RequirePackage[table]{xcolor}
\RequirePackage{xtab}
\RequirePackage{multirow}

% Links
\usepackage[colorlinks]{hyperref}
\hypersetup{
	allcolors = {red}
}

% Ecuaciones
\usepackage{amsmath}

% Rutas de fichero / paquete
\newcommand{\ruta}[1]{{\sffamily #1}}

% Párrafos
\nonzeroparskip


% Imagenes
\usepackage{graphicx}
\newcommand{\imagen}[2]{
	\begin{figure}[!h]
		\centering
		\includegraphics[width=0.9\textwidth]{#1}
		\caption{#2}\label{fig:#1}
	\end{figure}
	\FloatBarrier
}

\newcommand{\imagenflotante}[2]{
	\begin{figure}%[!h]
		\centering
		\includegraphics[width=0.9\textwidth]{#1}
		\caption{#2}\label{fig:#1}
	\end{figure}
}



% El comando \figura nos permite insertar figuras comodamente, y utilizando
% siempre el mismo formato. Los parametros son:
% 1 -> Porcentaje del ancho de página que ocupará la figura (de 0 a 1)
% 2 --> Fichero de la imagen
% 3 --> Texto a pie de imagen
% 4 --> Etiqueta (label) para referencias
% 5 --> Opciones que queramos pasarle al \includegraphics
% 6 --> Opciones de posicionamiento a pasarle a \begin{figure}
\newcommand{\figuraConPosicion}[6]{%
  \setlength{\anchoFloat}{#1\textwidth}%
  \addtolength{\anchoFloat}{-4\fboxsep}%
  \setlength{\anchoFigura}{\anchoFloat}%
  \begin{figure}[#6]
    \begin{center}%
      \Ovalbox{%
        \begin{minipage}{\anchoFloat}%
          \begin{center}%
            \includegraphics[width=\anchoFigura,#5]{#2}%
            \caption{#3}%
            \label{#4}%
          \end{center}%
        \end{minipage}
      }%
    \end{center}%
  \end{figure}%
}

%
% Comando para incluir imágenes en formato apaisado (sin marco).
\newcommand{\figuraApaisadaSinMarco}[5]{%
  \begin{figure}%
    \begin{center}%
    \includegraphics[angle=90,height=#1\textheight,#5]{#2}%
    \caption{#3}%
    \label{#4}%
    \end{center}%
  \end{figure}%
}
% Para las tablas
\newcommand{\otoprule}{\midrule [\heavyrulewidth]}
%
% Nuevo comando para tablas pequeñas (menos de una página).
\newcommand{\tablaSmall}[5]{%
 \begin{table}
  \begin{center}
   \rowcolors {2}{gray!35}{}
   \begin{tabular}{#2}
    \toprule
    #4
    \otoprule
    #5
    \bottomrule
   \end{tabular}
   \caption{#1}
   \label{tabla:#3}
  \end{center}
 \end{table}
}

%
% Nuevo comando para tablas pequeñas (menos de una página).
\newcommand{\tablaSmallSinColores}[5]{%
 \begin{table}[H]
  \begin{center}
   \begin{tabular}{#2}
    \toprule
    #4
    \otoprule
    #5
    \bottomrule
   \end{tabular}
   \caption{#1}
   \label{tabla:#3}
  \end{center}
 \end{table}
}

\newcommand{\tablaApaisadaSmall}[5]{%
\begin{landscape}
  \begin{table}
   \begin{center}
    \rowcolors {2}{gray!35}{}
    \begin{tabular}{#2}
     \toprule
     #4
     \otoprule
     #5
     \bottomrule
    \end{tabular}
    \caption{#1}
    \label{tabla:#3}
   \end{center}
  \end{table}
\end{landscape}
}

%
% Nuevo comando para tablas grandes con cabecera y filas alternas coloreadas en gris.
\newcommand{\tabla}[6]{%
  \begin{center}
    \tablefirsthead{
      \toprule
      #5
      \otoprule
    }
    \tablehead{
      \multicolumn{#3}{l}{\small\sl continúa desde la página anterior}\\
      \toprule
      #5
      \otoprule
    }
    \tabletail{
      \hline
      \multicolumn{#3}{r}{\small\sl continúa en la página siguiente}\\
    }
    \tablelasttail{
      \hline
    }
    \bottomcaption{#1}
    \rowcolors {2}{gray!35}{}
    \begin{xtabular}{#2}
      #6
      \bottomrule
    \end{xtabular}
    \label{tabla:#4}
  \end{center}
}

%
% Nuevo comando para tablas grandes con cabecera.
\newcommand{\tablaSinColores}[6]{%
  \begin{center}
    \tablefirsthead{
      \toprule
      #5
      \otoprule
    }
    \tablehead{
      \multicolumn{#3}{l}{\small\sl continúa desde la página anterior}\\
      \toprule
      #5
      \otoprule
    }
    \tabletail{
      \hline
      \multicolumn{#3}{r}{\small\sl continúa en la página siguiente}\\
    }
    \tablelasttail{
      \hline
    }
    \bottomcaption{#1}
    \begin{xtabular}{#2}
      #6
      \bottomrule
    \end{xtabular}
    \label{tabla:#4}
  \end{center}
}

%
% Nuevo comando para tablas grandes sin cabecera.
\newcommand{\tablaSinCabecera}[5]{%
  \begin{center}
    \tablefirsthead{
      \toprule
    }
    \tablehead{
      \multicolumn{#3}{l}{\small\sl continúa desde la página anterior}\\
      \hline
    }
    \tabletail{
      \hline
      \multicolumn{#3}{r}{\small\sl continúa en la página siguiente}\\
    }
    \tablelasttail{
      \hline
    }
    \bottomcaption{#1}
  \begin{xtabular}{#2}
    #5
   \bottomrule
  \end{xtabular}
  \label{tabla:#4}
  \end{center}
}



\definecolor{cgoLight}{HTML}{EEEEEE}
\definecolor{cgoExtralight}{HTML}{FFFFFF}

%
% Nuevo comando para tablas grandes sin cabecera.
\newcommand{\tablaSinCabeceraConBandas}[5]{%
  \begin{center}
    \tablefirsthead{
      \toprule
    }
    \tablehead{
      \multicolumn{#3}{l}{\small\sl continúa desde la página anterior}\\
      \hline
    }
    \tabletail{
      \hline
      \multicolumn{#3}{r}{\small\sl continúa en la página siguiente}\\
    }
    \tablelasttail{
      \hline
    }
    \bottomcaption{#1}
    \rowcolors[]{1}{cgoExtralight}{cgoLight}

  \begin{xtabular}{#2}
    #5
   \bottomrule
  \end{xtabular}
  \label{tabla:#4}
  \end{center}
}


















\graphicspath{ {./img/} }

% Capítulos
\chapterstyle{bianchi}
\newcommand{\capitulo}[2]{
	\setcounter{chapter}{#1}
	\setcounter{section}{0}
	\chapter*{#2}
	\addcontentsline{toc}{chapter}{#2}
	\markboth{#2}{#2}
}

% Apéndices
\renewcommand{\appendixname}{Apéndice}
\renewcommand*\cftappendixname{\appendixname}

\newcommand{\apendice}[1]{
	%\renewcommand{\thechapter}{A}
	\chapter{#1}
}

\renewcommand*\cftappendixname{\appendixname\ }

% Formato de portada
\makeatletter
\usepackage{xcolor}
\newcommand{\tutor}[1]{\def\@tutor{#1}}
\newcommand{\course}[1]{\def\@course{#1}}
\definecolor{cpardoBox}{HTML}{E6E6FF}
\def\maketitle{
  \null
  \thispagestyle{empty}
  % Cabecera ----------------
\noindent\includegraphics[width=\textwidth]{cabecera}\vspace{1cm}%
  \vfill
  % Título proyecto y escudo informática ----------------
  \colorbox{cpardoBox}{%
    \begin{minipage}{.8\textwidth}
      \vspace{.5cm}\Large
      \begin{center}
      \textbf{TFG del Grado en Ingeniería Informática}\vspace{.6cm}\\
      \textbf{\LARGE\@title{}}
      \end{center}
      \vspace{.2cm}
    \end{minipage}

  }%
  \hfill\begin{minipage}{.20\textwidth}
    \includegraphics[width=\textwidth]{escudoInfor}
  \end{minipage}
  \vfill
  % Datos de alumno, curso y tutores ------------------
  \begin{center}%
  {%
    \noindent\LARGE
    Presentado por \@author{}\\ 
    en Universidad de Burgos --- \@date{}\\
    Tutor: \@tutor{}\\
  }%
  \end{center}%
  \null
  \cleardoublepage
  }
\makeatother

\newcommand{\nombre}{Nombre del alumno} %%% cambio de comando

% Datos de portada
\title{título del TFG}
\author{\nombre}
\tutor{nombre tutor}
\date{\today}

\begin{document}

\maketitle


\newpage\null\thispagestyle{empty}\newpage


%%%%%%%%%%%%%%%%%%%%%%%%%%%%%%%%%%%%%%%%%%%%%%%%%%%%%%%%%%%%%%%%%%%%%%%%%%%%%%%%%%%%%%%%
\thispagestyle{empty}


\noindent\includegraphics[width=\textwidth]{cabecera}\vspace{1cm}

\noindent D. nombre tutor, profesor del departamento de nombre departamento, área de nombre área.

\noindent Expone:

\noindent Que el alumno D. \nombre, con DNI dni, ha realizado el Trabajo final de Grado en Ingeniería Informática titulado título de TFG. 

\noindent Y que dicho trabajo ha sido realizado por el alumno bajo la dirección del que suscribe, en virtud de lo cual se autoriza su presentación y defensa.

\begin{center} %\large
En Burgos, {\large \today}
\end{center}

\vfill\vfill\vfill

% Author and supervisor
\begin{minipage}{0.45\textwidth}
\begin{flushleft} %\large
Vº. Bº. del Tutor:\\[2cm]
D. nombre tutor
\end{flushleft}
\end{minipage}
\hfill
\begin{minipage}{0.45\textwidth}
\begin{flushleft} %\large
Vº. Bº. del co-tutor:\\[2cm]
D. nombre co-tutor
\end{flushleft}
\end{minipage}
\hfill

\vfill

% para casos con solo un tutor comentar lo anterior
% y descomentar lo siguiente
%Vº. Bº. del Tutor:\\[2cm]
%D. nombre tutor


\newpage\null\thispagestyle{empty}\newpage




\frontmatter

% Abstract en castellano
\renewcommand*\abstractname{Resumen}
\begin{abstract}
%En este primer apartado se hace una \textbf{breve} presentación del tema que se aborda en el proyecto.
La Minería de Datos es un campo de la informática que se ocupa de la obtención de información a partir de los datos dados. 
En los últimos años se ha encontrado con un enfoque más profundo debido a la implicación que tiene en los negocios, la economía o la política \cite{economist}.
Los factores clave con los que Data Mining está tratando son la inferencia de la información de los datos mediante la búsqueda y el hallazgo de patrones en los datos.
Teniendo esto en cuenta, se han desarrollado muchos algoritmos de reconocimiento de patrones para facilitar este proceso\cite{Witten:2011:DMP:1972514}.

La gran cantidad de datos, por otro lado, trae consigo dificultades a la hora de inferir información de los mismos. Sobre todo porque una gran cantidad de ella, al ser multietiquetada, trae consigo problemas en términos de tiempo y espacio a la hora de procesarla. Por lo tanto, se han propuesto algunas soluciones para hacer frente a este problema. Entre ellos, se encuentran los algoritmos de selección de características en los que se selecciona un cierto número de etiquetas teniendo en cuenta la relevancia que tienen para las instancias de datos.

Por lo tanto, en este proyecto algunos de estos algoritmos han sido implementados en la biblioteca de Sklearn ml-sklearn utilizando el lenguaje python. Para ello se ha utilizado la guía python pep y el sklearn. Para la experiencia más práctica del usuario, se han construido algunos portátiles, en los que se han utilizado diferentes conjuntos de datos. Junto con estos, también se han construido algunos gráficos para que se vea la eficacia. Los algoritmos construidos son los siguientes: Binary Relevance, Label Powerset y Label Construction for Feature Selection\cite{Spolaor:2016:SRM:2895226.2895397}.

\end{abstract}

\renewcommand*\abstractname{Descriptores}
\begin{abstract}
Multi-label, Feature Selection, Binary Relevance, Label Powerset, Data Mining \ldots
\end{abstract}

\clearpage

% Abstract en inglés
\renewcommand*\abstractname{Abstract}
\begin{abstract}
%A \textbf{brief} presentation of the topic addressed in the project.
Data Mining is a field in computer science which deals with the gaining of information out of the data given. 
In recent years has encountered a more deep approach because of the implication it has in various fields in which business, economy or politics can be included\cite{economist}.
The key factors that Data Mining is dealing with are the inference of information from data by searching and finding patterns in the data.
With this taken into account, a lot of pattern recognition algorithms have been developed in order to facilitate this process\cite{Witten:2011:DMP:1972514}.

The big amount of data, on the other hand, brings difficulties when it comes to infer information out of data. Specifically, this problem arouses when the data to be mined is multi-labeled\cite{multi-label}. This brings issues in terms of time and space during the process. Therefore, some solutions have been proposed in order to handle this issue. Among which, is the Feature Selection methods in which a certain number of labels are selected taking into account the relevance they have for the data instances\cite{featureSelectionPereira}.

Therefore in this project some of these algorithms have been implemented over the Sklearn (Scikit-Learn)\cite{scikitlearn} and scikit-multilearn library\cite{skml}, by using the Python language\cite{python}. The Python style guide (PeP~\cite{pep} and the guide for Sklearn have been used in doing so. For the more hand on experience of the user, some notebooks have been build, in which different datasets have been used. Along with these, some graphics have been also plotted in order for the efficacy to be seen. The algorithms constructed are the following: Binary Relevance\cite{Zhang2018}, Label Powerset\cite{featureSelectionPereira}, RAndom k-labELsets (RAKEL)\cite{RAKEL} and Label Construction for feature Selection\cite{LabelConstruction}. 
\end{abstract}

\renewcommand*\abstractname{Keywords}
\begin{abstract}
Multi-label, Feature Selection, Binary Relevance, Label Powerset, Data Mining, RAndom k-labELsets,
\end{abstract}

\clearpage

% Indices
\tableofcontents

\clearpage

\listoffigures

\clearpage

\listoftables
\clearpage

\mainmatter
\capitulo{1}{Introduction}

Data Mining is a field in computer science which deals with the gain of information out of the data given. 
In recent years has encountered a more deep approach because of the implication it has in various fields in which business, economy or politics can be included\cite{economist}.
Using different methods from the field of machine learning, statistics or database systems, through the process of data mining patterns in the data sets are discovered.
The key factors that Data Mining is dealing with are the inference of information from data by searching and finding patterns in the data. After the inference or extraction of the information, the phase of making this information comprehensible for different uses takes place\cite{datamining}. Data mining is also the analysis step in "knowledge discovery in databases" process (KDD)\cite{KDD}.
With this taken into account, a lot of pattern recognition algorithms have been developed in order to facilitate this process\cite{Witten:2011:DMP:1972514}.

multi label, single label and feature selecting algorithms

the data given can sometimes be single label wich turn to be a single class clasification problem. for example, when it comes to classicifation the algorithm would predict if the instance to be predicted is or not of a specific class, which turns this problem into a binary clasification problem. Example, if its a dog or not in an image.

But, as for many real worl applications, usually the data is not single labeld which leads to the complication of the problem. Multi labeled data can have more labels that have to be classified for each instance. For instance, an image can have as label more things, sand, water, palm trees, sun, sky, chair, becnh. In order for an algorith to classifie an image from this kind of data set, it should take all the labels into account in order to do so.

This is when feature selection methods come in hand, because some labels can be relevant to the classification process and others no, when it comes to the classification. In our case, the labels sand, water and palm trees cand be very relevant, as opossed to the ones that may not be so relevant, as 'bench', or chair. Methods to do this is by binary relevance or label powerset.

feature selection






\include{./tex/2_Objetivos_del_proyecto}
\include{./tex/3_Conceptos_teoricos}
\include{./tex/4_Tecnicas_y_herramientas}
\include{./tex/5_Aspectos_relevantes_del_desarrollo_del_proyecto}
\include{./tex/6_Trabajos_relacionados}
\include{./tex/7_Conclusiones_Lineas_de_trabajo_futuras}


\bibliographystyle{plain}
\bibliography{bibliografia}

\end{document}
