\capitulo{1}{Introduction}

Data Mining is a field in computer science which deals with the gain of information out of the data given. 
In recent years has encountered a more deep approach because of the implication it has in various fields in which business, economy or politics can be included\cite{economist}.
Using different methods from the field of machine learning, statistics or database systems, through the process of data mining patterns in the data sets are discovered.
The key factors that Data Mining is dealing with are the inference of information from data by searching and finding patterns in the data. After the inference or extraction of the information, the phase of making this information comprehensible for different uses takes place\cite{datamining}. Data mining is also the analysis step in "knowledge discovery in databases" process (KDD)\cite{KDD}.
With this taken into account, a lot of pattern recognition algorithms have been developed in order to facilitate this process\cite{Witten:2011:DMP:1972514}.

multi label, single label and feature selecting algorithms

the data given can sometimes be single label wich turn to be a single class clasification problem. for example, when it comes to classicifation the algorithm would predict if the instance to be predicted is or not of a specific class, which turns this problem into a binary clasification problem. Example, if its a dog or not in an image.

But, as for many real worl applications, usually the data is not single labeld which leads to the complication of the problem. Multi labeled data can have more labels that have to be classified for each instance. For instance, an image can have as label more things, sand, water, palm trees, sun, sky, chair, becnh. In order for an algorith to classifie an image from this kind of data set, it should take all the labels into account in order to do so.

This is when feature selection methods come in hand, because some labels can be relevant to the classification process and others no, when it comes to the classification. In our case, the labels sand, water and palm trees cand be very relevant, as opossed to the ones that may not be so relevant, as 'bench', or chair. Methods to do this is by binary relevance or label powerset.

feature selection





